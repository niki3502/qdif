\documentclass[a4paper, 14pt]{extarticle}

\usepackage{textcomp}
\usepackage{setspace}
\onehalfspacing


%basic geometry of paper
\usepackage[T2A]{fontenc}
\usepackage[utf8]{inputenc}
\usepackage[english,russian]{babel}
\usepackage[mag=1000,a4paper,left=2.2cm,right=3.0cm,top=3.1cm,bottom=3.0cm]{geometry}


%Math
\usepackage{amsmath,amsfonts,amssymb,amsthm,mathtools} % AMS
\usepackage[italicdiff]{physics}
\usepackage{euscript} %Для эпсилон
\usepackage{dsfont} %Для знака вещественного пространства
\usepackage{mathtext} 	
\usepackage{icomma} % "Умная" запятая: $0,2$ --- число, $0, 2$ --- перечисление
\usepackage{stmaryrd}

\DeclareMathOperator*{\argmax}{arg\,max}
\DeclareMathOperator*{\val}{val}
\let\phi\varphi
\let\uglyepsilon\epsilon
\let\epsilon\varepsilon
\let\kappa\varkappa
\theoremstyle{definition}
% \theoremstyle{plain}
\newtheorem{theorem}{Theorem}
\newtheorem{lemma}{Lemma}

\theoremstyle{remark}
\newtheorem{remark}{Remark}

\theoremstyle{definition}
\newtheorem{definition}{Definition}
\newtheorem{example}{Example}

%vector graphs
\usepackage{tikz}
\usepackage{pgf}


%bibliography
\usepackage[backend=biber,bibencoding=utf8,sorting=none,style=ieee, citestyle=numeric-comp]{biblatex}
\usepackage{csquotes}


\usepackage{mathtools}  
\mathtoolsset{showonlyrefs}  
%referencing
% \usepackage{cleveref}

%big comments
\usepackage{comment}

\addbibresource{literature.bib}
\author{Гаянов Никита Владимирович}
\title{$q$-разностные уравнения}
\date{\today}
% \AtBeginBibliography{\Large}
\begin{document}

%\input{sections/titlepage}
%\input{sections/intro}

\maketitle

\begin{abstract}
    This is a draft about $q$-difference equations and asymptotic expansions of their solutions. Here you might see theory about linear $q$-difference equations, power expansions, and new approach for searching logarithmic solution.
\end{abstract}

\section{Preliminary}
What is $q$-difference? This is a little bit tricky question. First aff all we would need to define $q$-different operator $\sigma_q$(usually we well use only one $q$ in equation and omit it), which is actually is family of operators, depending on $q$. And domain of such an operator may be really different. There are some examples.

Consider all functions acting from $\mathbb{C}$ to subset of $\mathbb{C}$.
Then $\sigma_q [f](z) = f(qz)$.

Consider Hahn series $\mathbb{C} \llbracket z^{\mathbb{R}} \rrbracket$.
A Hahn series $f$ with complex coefficients and
real exponents is a formal sum $f(z) =\sum_{\mu \in \mathbb{R}} f_\mu z^\mu$ whose coefficients
$f_\mu$ are complex numbers and whose support, that is, the set
$\{ \mu \in \mathbb{R} : f_\mu \neq 0 \}$ ,
is a well ordered set.

For such Hahn series, $\sigma$ is

$\sigma[f](z) = \sum_{\mu \in \mathbb{R}} f_\mu q^\mu z^\mu$. Note that Hahn series needn't to converge anywhere.


And for sake of power-log expansions consider power-log transeries:
power log-transeries $f$ is $\sum_{i,j=0}^\infty f_{ij}\log_q^j(x) z^i$.

Here we define
$\sigma[f](z) = \sum_{i,j=0}^\infty f_{ij}(\log_q(z)+1)^j z^i$.

Note that here definition is a a little bit trickier than in previous, because logarithm is a branching function, and $\sigma\log_q(z)$ not always equals $1+\log_q(z)$.
Also you may see, that if $q=1$, then we actually don't have $q$-diffirences.
Also, we don't use $q=0$, since $\sigma[f](x) \equiv f(0)$.

So, we start from strict definitions.

\begin{definition}[Dulac series]
Dulac series is a formal series of the form $\sum_{k=0}^\infty P_k(\log_q z)z^k$, where $P_k$ is a polynomial.    
\end{definition}

\begin{definition}[Equal Dulac series]
Two Dulac serias are formally equal ($\sum_{k=0}^\infty P_k(\log_q z)z^k$ and $\sum_{k=0}^\infty Q_k(\log_q z)z^k$), if $P_k \equiv Q_k$ for all $k\in Z_+$.    
\end{definition}

For the rest of paper we will consider only formal equality of Dulac series, so we will use a different designations.

\begin{definition}[Dulac series - 2]
Dulac series is a formal series of two independent variables of the form $\sum_{k=0}^\infty P_k(t)z^k$, where $P_k$ is a polynomial.    
\end{definition}

\begin{remark}
    You can see, that two trans-series are equial in values
    $z = 1 + \sum_{k=1}^\infty \frac{\ln^k z}{k!}, if |z-1|<1$, but they are not equal in the formal meaning.
\end{remark}

If we look at $q$-difference for actual function. You see that
$\sigma[z] = qz$, if we look at function $w = \log_q z = \frac{\ln_z}{\ln_q}$, where $\ln_z$ is defined in all complex plane exept the set $\mathbb{R}_-$, and $\log_q 1 = 0$, then you may see that $\log_q z = \log_q z + 1$ if $ -\pi<\arg z + \arg q \leq \pi $. If inequality does not hold, then $|\log_q qz| = |\log_q z + 1|$, $\arg \log_qz = \arg(\log_q z + 1) + 2\pi n$ for some $n\in \mathbb{Z}$ such that $-\pi<\arg(\log_q z + 1) + 2\pi n \leq \pi$.

\begin{definition}[$q$-difference operator for Dulac series]
Let $f = \sum_{k=0}^\infty P_k(t)z^k$, then $\sigma[f] =  \sum_{k=0}^\infty P_k(t+1)q^kz^k$ is a $q$-difference derivative of $f$.
\end{definition}



\begin{definition}
    An algebraic $q$-difference equation is equation of the form
    \begin{equation}
        y
    \end{equation}
\end{definition}

Для такого класса уравнений существуют алгоритмы нахождения степенных разложений (см. cite), однако известно, что существуют нестепенные разложения, например уравнение
$$y(qx) - y(x) = 1$$
обладает решением $y(x) = \log_q x$ (главная ветвь), которое не представляется разложением степенного ряда в окрестности(в некотором секторе) нуля.
Действительно:
\begin{multline*}
    y(qx) - y(x) = \log_q (qx) - \log_q x = \frac{\ln |qx| - \ln|x| + i\arg (qx) - i\arg (x)}{\ln|q| + i\arg(q)} =\\
    = \frac{\ln|q| + i\arg (qx) - i\arg (x)}{\ln|q| + i\arg(q)} = 1
\end{multline*}
при условии, что $\arg (qx) = \arg(q) + \arg(x)$.
Для этого необходимо и достаточно, чтобы $-\pi < \arg(q) + \arg(x) \leqslant \pi$, т.е
$y(x) = \log_q x$, является решением уравнения $y(qx) - y(x) = 1$ в секторе $-\pi - \arg(q) < \arg(x) \leqslant \pi - \arg(q)$



\begin{theorem}
    Уравнение вида
    \begin{equation} L(\sigma)y = x M(x, \log_q x, y, \sigma y, \ldots, \sigma^n y),\label{eq1}\end{equation}
    где $L(\sigma) = \sum_{j = 0}^n a_j(\sigma)^j $ -- многочлен от оператора $\sigma$,
    
    $M\in \mathbb{C}[x, t, y_0, \ldots, y_n]$ -- многочлен от $n+3$ переменных,
    
    обладает $N$-параметрическим формальным решением уравнения \eqref{eq1} в виде ряда $\sum_{k=0}^\infty P_k(\log_q x)x^k$, где $P_k$ -- многочлены, 
    $N$ -- количество корней  $L(z)$, которые имеют вид  $q^n, n \in \mathbb{N}$.
\end{theorem}

\begin{remark}
    Если $N=0$, то говорим, что отсутствует резонанс.
\end{remark}

\begin{remark}
    Если $N=0$, а в правой части отсутствуют логарифмы, то уравнение обладает степенным решением.
\end{remark}
\begin{remark}
    Ряд $\sum_{k=0}^\infty P_k(\log_q x)x^k$ называется рядом Дюлака.
\end{remark}

\begin{example}[резонанса нет и в правой части нет логарифма, решение выражается в виде степенного ряда(многочлена)]
    \begin{gather*}
    \sigma y - 2 y = q x,\\
    y(x) =
    \begin{cases}
        \frac{q}{q-2} x, q\neq 2;\\
        x \log_2 x, q = 2.
    \end{cases}\\
    \end{gather*}
\end{example}


\begin{example}[резонанса нет и в правой части логарифм, решение выражается в виде ряда]
\begin{gather*}
\sigma y - (q+2) y = x \log_q x + x y,\\ 
y(x) = -\left(\frac{1}{2}\log_q x + q/4\right)x + o(x)\,, x\to 0.\\
\end{gather*}
\end{example}

\begin{example}[резонанса нет, справа логарифм, можно посчитать решение]
    \begin{gather*}
    \sigma y - 2 y = x\log_q x,\\
    q P_1(t+1) - 2P_1(t) = t,\\
    y(x) = \begin{cases}
        x\left(\frac{\log_q x}{q-2} + \frac{q}{(q-2)^2}  \right), q\neq 2 \\
        x \left( \log^2_q x/4 - 1/4 \log_q x + C \right), q = 2.
    \end{cases} 
    \end{gather*}
\end{example}



\begin{example}[резонанса нет, справа логарифм, можно посчитать решение]
    \begin{gather*}
    \sigma y - 2 y = x\log_q x,\\
    y(x) = x\left(\frac{\log_q x}{q-2} + \frac{q}{(q-2)^2}  \right), q\neq 2.
    \end{gather*}
\end{example}

 \begin{example}[Есть резонанс, в правой части нет логарифма, есть решение в виде ряда Дюлака]
    \begin{gather*}
    \sigma y - q y = x,\\
    y(x) = (C + \log_q x)\frac{x}{q}.
    \end{gather*}
\end{example}

Уравнение \eqref{eq1} не является алгебраическим, так как в правой части присутствует логарифм.
Однако была доказано, что при существовании решения в виде ряда Дюлака для некоторые алгебраические $q$-разностные уравнения сводятся заменой к виду \eqref{eq1}.

\begin{theorem}
    Пусть $q$-разностное алгебраическое уравнение
       \eqref{eq0} 

        обладает формальным решением в виде ряда Дюлака.
        
    $$F(x, \Phi) = 0,
        \Phi = (\varphi, \sigma \varphi, \ldots, \sigma^n \varphi).
       $$
        Пусть также
        $
        \frac{\partial F}{\partial y_j}(x,\Phi) = a_j x^m + b_j(\log_q x)x^{m+1} + \ldots \notag
        $
    и пусть не все $a_j=0$. Тогда существует замена, которая приводит уравнение к специальному виду \eqref{eq1}.
    \end{theorem}

    \begin{example}
    Уравнение
    \[\sigma y- qx - y^2 = 0\]
    сводится заменой $y = 1 + \frac{q}{q-2}x + xv$
    к виду 
    \[q \sigma v = x\left(\left( \frac{q}{q-2} \right)^2 + 2 v \frac{q}{q-2} + v^2 \right). \]
    \end{example}

    Для нахождения начальных членов разложеня предлагается использовать метод многоугольников Ньютона: 

    \begin{definition}
        Для алгебраического уравнения
        $$F(x, y, \sigma y, \ldots, \sigma^n y) = \sum_{k=1}^n C_k x^{a_k} y^{\alpha_{k0}} \ldots (\sigma^n y)^{\alpha_{kn}}=0$$ 

        многоугольником Ньютона называется
        $$N(F) = Conv \left( \left\{ \left(a_k, \sum_{i=0}^n \alpha_{ki}\right)\right\}_{k=1}^N\right),$$
        то есть каждому моному $F$ ставится в соответствие точка на плоскости. Выпуклая оболочка этих точек прендставляет собой многоугольнгик Ньютона.
    \end{definition}

    \begin{lemma}
        Степенными преобразованиями:
        \begin{enumerate}
            \item $y = x^s \hat{y},$
            \item $F (x, y, \ldots, \sigma^n y) = x^s \hat{F} (x, y, \ldots, \sigma^n y)$.
        \end{enumerate}
Можно привести любое негоризонтальное ребро в
многоугольнике Ньютона к вертикальному.
    \end{lemma}

    Вертикальным ребрам соответствует укороченное уравнение
    \begin{equation}
    G(P(t),\ldots, P(t+n)) = \sum_k \gamma_k q^\lambda P^{\alpha_{k_0}}(t)\ldots P^{\alpha_{k_n}}(t+n)= 0. \label{eq:3}\end{equation}
    Это алгебраическое разностное автономное уравнение.
    Для данного уравнения ищются решения в виде многочленов.

    Как было показано в (cite), если $n=1$, тоб если решение в виде многочлена $P(t)$ существует, то $\deg P \leq \deg G$.
    В случае $n>1$ существуют уравнения, для которых не будет верхней оценки степени решения.

    \begin{example}
        \begin{multline}
    G(P(x), P(x-1), P(x-2), P(x-3)) = P(x)P(x-2)P(x-3) -\\- 2P(x-1)^2 P(x-3) + P(x-1)P(x-2)^2 +\\
    + P(x)P(x-1)P(x-3) - 2P(x)P(x-2)^2 + P(x-1)^2 P(x-2) = 0
\end{multline}
имеет решение в виде факториальной степени:

$g_n(x) = (x+1)(x+a-1)\ldots(x+a-(n-1))$.

    \end{example}
\end{document}