\usepackage{textcomp}
\usepackage{setspace}
\onehalfspacing


%basic geometry of paper
\usepackage[T2A]{fontenc}
\usepackage[utf8]{inputenc}
\usepackage[english,russian]{babel}
\usepackage[mag=1000,a4paper,left=2.2cm,right=3.0cm,top=3.1cm,bottom=3.0cm]{geometry}


%Math
\usepackage{amsmath,amsfonts,amssymb,amsthm,mathtools} % AMS
\usepackage[italicdiff]{physics}
\usepackage{euscript} %Для эпсилон
\usepackage{dsfont} %Для знака вещественного пространства
\usepackage{mathtext} 	
\usepackage{icomma} % "Умная" запятая: $0,2$ --- число, $0, 2$ --- перечисление
\usepackage{stmaryrd}

\DeclareMathOperator*{\argmax}{arg\,max}
\DeclareMathOperator*{\val}{val}
\let\phi\varphi
\let\uglyepsilon\epsilon
\let\epsilon\varepsilon
\let\kappa\varkappa
\theoremstyle{definition}
% \theoremstyle{plain}
\newtheorem{theorem}{Theorem}
\newtheorem{lemma}{Lemma}

\theoremstyle{remark}
\newtheorem{remark}{Remark}

\theoremstyle{definition}
\newtheorem{definition}{Definition}
\newtheorem{example}{Example}

%vector graphs
\usepackage{tikz}
\usepackage{pgf}


%bibliography
\usepackage[backend=biber,bibencoding=utf8,sorting=none,style=ieee, citestyle=numeric-comp]{biblatex}
\usepackage{csquotes}


\usepackage{mathtools}  
\mathtoolsset{showonlyrefs}  
%referencing
% \usepackage{cleveref}

%big comments
\usepackage{comment}